\documentclass{article}

\title{PhD Outline}
\author{Wellen}
\date{\today}

\usepackage[utf8]{inputenc}
\usepackage[margin=1in]{geometry}
\usepackage{url, appendix, pdfpages, subcaption} %for inputting urls, appendices, outside docs, and placing multiple pictures in the same figure
%http://blog.bharatbhole.com/inserting-pages-from-an-external-pdf-document-within-a-latex-document/
\usepackage{float} %[H] tells LaTeX you absolutely want an image there
\usepackage{fancyhdr, enumitem}
\newcommand{\compresslist}{ % Define a command to reduce spacing within itemize/enumerate environments, this is used right after \begin{itemize} or \begin{enumerate}
	\setlength{\itemsep}{0pt}
	\setlength{\parskip}{0pt}
	\setlength{\parsep}{10pt}
}
\usepackage{amsmath, amssymb, mathrsfs, dsfont, amsthm,  cancel}
\usepackage[mathscr]{eucal} %for getting less bold and swirlier math scripts
\usepackage{graphicx, tikz, wrapfig} %for figures, figure creation, and haveing text to the left of a figure
\allowdisplaybreaks[1]

\usepackage{titlesec}
%\titlespacing*{\subsubsection}{0pt}{0\baselineskip}{\baselineskip}

%Places the title information in the header
\makeatletter         
%\def\@maketitle{{\bfseries\@title}\\ \@author \\ \@date }


%%%%%%%%%%%%%%%%%%%%%%%%%%%%%%%%%%%%%%%%%%%%%%%%%%%%%%%%%%%%%%%%%%%%%%
%
%								MACROS
%
%%%%%%%%%%%%%%%%%%%%%%%%%%%%%%%%%%%%%%%%%%%%%%%%%%%%%%%%%%%%%%%%%%%%%%

%                   Brackets and Parenthesis

\newcommand{\<}{\langle}
\renewcommand{\>}{\rangle}
\renewcommand{\(}{\left(}
\renewcommand{\)}{\right)}
\renewcommand{\[}{\left[}
\renewcommand{\]}{\right]}

%                   Math Blackboard Bold Symbols

\newcommand\Cb{\mathds{C}}
\newcommand\Eb{\mathds{E}}
\newcommand\Fb{\mathds{F}}
\newcommand\Gb{\mathds{G}}
\newcommand\Pb{\mathds{P}}
\newcommand\Qb{\mathds{Q}}
\newcommand\Rb{\mathds{R}}
\newcommand\Zb{\mathds{Z}}
\newcommand\Nb{\mathds{N}}
\newcommand\Vb{\mathds{V}}

%                   mathscr symbols

\newcommand\Ac{\mathscr{A}}
\newcommand\Bc{\mathscr{B}}
\newcommand\Cc{\mathscr{C}}
\newcommand\Dc{\mathscr{D}}
\newcommand\Ec{\mathscr{E}}
\newcommand\Fc{\mathscr{F}}
\newcommand\Gc{\mathscr{G}}
\newcommand\Hc{\mathscr{H}}
\newcommand\Lc{\mathscr{L}}
\newcommand\Mc{\mathscr{M}}
\newcommand\Nc{\mathscr{N}}
\newcommand\Oc{\mathscr{O}}
\newcommand\Pc{\mathscr{P}}
\newcommand\Qc{\mathscr{Q}}
\newcommand\Sc{\mathscr{S}}
\newcommand\Kc{\mathscr{K}}
\newcommand\Jc{\mathscr{J}}
\newcommand\Xc{\mathscr{X}}
\newcommand\Yc{\mathscr{Y}}
\newcommand\Zc{\mathscr{Z}}

%                   shortcuts for greek letters

\newcommand\eps{\epsilon}
\newcommand\om{\omega}
\newcommand\Om{\Omega}
\newcommand\sig{\sigma}
\newcommand\Sig{\Sigma}
\newcommand\Lam{\Lambda}
\newcommand\gam{\gamma}
\newcommand\Gam{\Gamma}
\newcommand\lam{\lambda}
\newcommand\del{\delta}
\newcommand\Del{\Delta}
\newcommand\Chi{\mathcal{X}}

%                   Letters with bars

\newcommand\Wb{\overline{W}}
\newcommand\Mb{\overline{M}}
\newcommand\Xb{\overline{X}}
\newcommand\Yb{\overline{Y}}
\newcommand\Sb{\overline{S}}
\newcommand\Pbb{\overline{\Pb}}
\newcommand\Ebb{\overline{\Eb}}
\newcommand\Acb{\bar{\Ac}}
\newcommand\ab{\overline{a}}
\newcommand\bb{\overline{b}}
\newcommand\hb{\overline{h}}
\newcommand\xb{\bar{x}}
\newcommand\yb{\bar{y}}
\newcommand\zb{\bar{z}}
\newcommand\Ab{\bar{\Ac}}
\newcommand\vb{\bar{v}}
\newcommand\ub{\bar{u}}
\newcommand\qb{\bar{q}}
\newcommand\pb{\bar{p}}
%\newcommand\Vb{\bar{V}}
\newcommand\rhob{\bar{\rho}}
\newcommand\IIb{\bar{\II}}
\newcommand\LLb{\bar{\LL}}
\newcommand{\psib}{\bar{\psi}}
\newcommand{\etab}{\bar{\eta}}
\newcommand{\xib}{\bar{\xi}}

%                   Letters with underlines

\newcommand\Wu{\underline{W}}
\newcommand\Xu{\underline{X}}
\newcommand\Mu{\underline{M}}

%                   Vectors (bolded)

\newcommand\Av{\mathbf{A}}
\newcommand\Fv{\mathbf{F}}
\newcommand\xv{\mathbf{x}}
\newcommand\bv{\mathbf{b}}
\newcommand\pv{\mathbf{p}}
\newcommand\qv{\mathbf{q}}
\newcommand\ev{\mathbf{e}}
\newcommand\yv{\mathbf{y}}
\newcommand\zv{\mathbf{z}}
\newcommand\Yv{\mathbf{Y}}
\newcommand\Hv{\mathbf{H}}
\newcommand\Cv{\mathbf{C}}
\newcommand\mv{\mathbf{m}}
\newcommand\etav{\boldsymbol\eta}
\newcommand\piv{\boldsymbol \pi}
\newcommand\muv{\boldsymbol \mu}
\newcommand\Sv{\textbf S}
\newcommand\Qv{\textbf Q}
\newcommand\zev{\textbf 0}

%                   Letters with Hats

\newcommand\Pbh{\widehat{\Pb}}
\newcommand\Ebh{\widehat{\Eb}}
\newcommand\Qh{\widehat{Q}}
\newcommand\Ih{\widehat{I}}
\newcommand\pih{\widehat{\pi}}
\newcommand\Pih{\widehat{\Pi}}
\newcommand\varphih{\widehat{\varphi}}
\newcommand\Wh{\widehat{W}}
\newcommand\Fh{\widehat{F}}
\newcommand\Yh{\widehat{Y}}
\newcommand\Ah{\widehat{\Ac}}
\newcommand\uh{\widehat{u}}
\newcommand\Uh{\widehat{U}}
\newcommand\vh{\widehat{v}}
\newcommand\fh{\widehat{f}}
\newcommand\Hh{\widehat{H}}
\newcommand\hh{\widehat{h}}
\newcommand\Bh{\widehat{B}}
\newcommand\etah{\widehat{\eta}}
\newcommand\Lh{\widehat{L}}

%                   Letters with Tildes

\newcommand\Ebt{\widetilde{\Eb}}
\newcommand\Pbt{\widetilde{\Pb}}
\newcommand\Act{\widetilde{\Ac}}
\newcommand\Lct{\widetilde{\Lc}}
\newcommand\Gct{\widetilde{\Gc}}
\newcommand\Xct{\widetilde{\Xc}}
\newcommand\Yct{\widetilde{\Yc}}
\newcommand\Zct{\widetilde{\Zc}}
\newcommand\Mt{\widetilde{M}}
\newcommand\Wt{\widetilde{W}}
\newcommand\Bt{\widetilde{B}}
\newcommand\Nt{\widetilde{N}}
\newcommand\Xt{\widetilde{X}}
\newcommand\xt{\widetilde{x}}
\newcommand\ut{\widetilde{u}}
\newcommand\kappat{\widetilde{\kappa}}
\newcommand\at{\widetilde{a}}
\newcommand\Gamt{\widetilde{\Gam}}
\newcommand\Pct{\widetilde{\Pc}}


%         Theorem environments

\newtheoremstyle{mine}% name of the style to be used
  { }% measure of space to leave above the theorem. E.g.: 3pt
  { }% measure of space to leave below the theorem. E.g.: 3pt
  { }% name of font to use in the body of the theorem
  { }% measure of space to indent
  { }% name of head font
  { }% punctuation between head and body
  { }% space after theorem head; " " = normal interword space
  {\thmname{\textbf{#1}}\thmnumber{\textbf{ #2}}:\thmnote{ #3}\\}% Manually specify head

\theoremstyle{mine}
\newtheorem{thm}{Theorem}[section]
\newtheorem{cor}[thm]{Corollary}
\newtheorem{prop}[thm]{Proposition}
\newtheorem{defn}[thm]{Definition}
\newtheorem{rmk}[thm]{Remark}
\newtheorem{lem}[thm]{Lemma}
\newtheorem{assumption}[thm]{Assumption}


%                   other macros

\newcommand{\dd}{\partial}
\newcommand{\ind}{\perp \! \! \! \perp}
\newcommand\ii{\mathtt{i}}
\renewcommand\d{\mathrm{d}}
\newcommand\ee{\mathrm{e}}
\newcommand\BS{\textrm{BS}}
\newcommand\ko{\mathrm{ko}}
\newcommand\ki{\mathrm{ki}}
\newcommand\rb{\mathrm{rb}}
\newcommand\eu{\mathrm{eu}}

%          other


\newcommand\Ib{\mathds{1}}
\renewcommand\Re{\operatorname{Re}}
\renewcommand\Im{\operatorname{Im}}
\newcommand{\nuba}[1]{\overline{\nu_{#1}^\ast}}
\newcommand{\tab}{\hspace{.4 in}}
\newcommand{\enter}{\vspace{.15 in}}
\newcommand{\fa}{\forall}
\newcommand{\tf}{\therefore}
\newcommand{\indep}{\raisebox{0.05em}{\rotatebox[origin=c]{90}{$\models$}}}
\newcommand{\nth}[2]{#1^{\text{\tiny #2}}}
\newcommand{\xRightarrow}[2][]{\ext@arrow 0359\Rightarrowfill@{#1}{#2}}
\newcommand{\myeq}[1]{\mathrel{\overset{\makebox[0pt]{\mbox{\small {#1}}}}{=}}}
\newcommand{\since}{\reflectbox{\rotatebox[origin=c]{180}{$\therefore$}}}
\makeatother


\begin{document}

\title

\section{Generic Model from Literature}

	A Simple Dynamical System - \underline{Economic Dynamics: Phase Diagrams and Their Economic Application} by Ronald Shone
	
	\begin{align}
		q_d =& a - bp && b > 0 \label{eq:qd1}\\ 
		q_s =& c + fp && f > 0 \label{eq:qp1}\\ 
		\frac{\d p}{\d t} =& \alpha(q_d - q_s) = \alpha(b+f)p - \alpha(a-c) && \alpha > 0 \label{eq:pt1}
	\end{align}
	There is an equilibrium at p(t) = \frac{a-c}{b+f} + \[p_0 - \(\frac{a-c}{b+f}\)\]e^{-\alpha(b+f)t}
	%p(0) = p_0
	$q_d,\ q_s,\ \&\ p$ are continuous functions of time.
	
	Analyzing the Jacobian of this system for $q_d$ and $p$ we get
	$$\begin{matrix}
		0 & -b\\
		0 & \alpha(b+f)
	\end{matrix}.$$
	From here I calculated the eigenvalues to find $\lambda_1 = \alpha(b+f)$ and $\lambda_2 = 0$. $\lamba_2 = 0$ could be a problem if we were dealing with a nonlinear system, however this is a linear system so the behavior of the fixed points is determined by $\lambda_1$. This means that if $\lambda_1 < 0 $ the system is stable, and if $\lambda_1 > 0$ then it is unstable. However we know that $\lambda_1 > 0 $, since the restraints on our system require that $\alpha>0$, $b>0$, and $f>0$. For $b$ and $f$, this is to account for how we know demand decreases and supply increases as price increases, which are well established results in economics except in extreme outliers. $\alpha >0$ comes from examining $(q_d - q_s)$. If $\alpha < 0$, then we could rewrite it as $|\alpha|(q_s - q_d)$, but demand does not shift to meet supply, supply and price both shift to meet demand in an economy. Thus we see that for all possible versions of this system, we have an unstable equilibrium.  
	
	\begin{figure}[b]
		\centering
			\includegraphics[width = 0.6\columnwidth]{Figures/simple_system.png}
		\caption{The vector field plot of the system described in \eqref{eq:qd1}}
		\label{fig:model1} 
	\end{figure}

Numerically I analyzed the system with $a = 10,\ b = 1,\ c = 0,\ f = 0.5,$ and $alpha = 0.5$. The orange line is the quantity demanded that I assumed in equation \eqref{eq:qd1}. The green line is the quantity supplied that I assumed in equation \eqref{eq:qs1}. Consumer utility for the product is assumed to not change over time, and so these demand lines are both held constant over time. We see in Figure \ref{fig:model1} the equilibrium point of $(6 \frac{2}{3}, 3 \frac{1}{3})$ in the classic supply and demand system, and also how for this system a price of $6 \frac{2}{3}$ does not change. Further, the quantity demanded can also be thought of as the maximum quantity sold at any given price, so only the dynamics in the bottom left half of the plot should be analyzed.

As expected from the analysis, in figure \ref{fig:model1} we see that prices do not flow towards the price equilibrium, but instead to move away from it. This is made even more clear in figure \ref{fig:model1_time_price}, which is the time evolution of price. In this system we see that the price can only reach the equilibrium of $6 \frac{2}{3}$ by already starting at that value. This is in direct conflict to current theory which states that prices will converge to equilibrium. To achieve a dynamical system that fits a stable equilibrium, we must break one of our main underlying economic assumptions.

There are many problems with this model. First the dynamics don't make sense. Then, this model does not indicate at all how things evolve if the demand or supply shifts.  With the underlying issues in the model, it is not clear that any conclusions drawn would be meaningful. 

	\begin{figure}
		\centering
			\includegraphics[width = 0.6\columnwidth]{Figures/simple_priceOverTime.png}
		\caption{The evolution of price over time according to \eqref{eq:pt1}}
		\label{fig:model1_time_price} 
	\end{figure}



%%%%%%%%%%%%%%%%%%%%%%%%%%%%%%%%%%%%%%%%%%%%%%%%%%%%%%%%%%%%%%%%%%%%%%%%%%%%%%%%%%%%%%%%%%%%%%%%%%%%%%%%%%%%%%%%%%%%%%%%%%%%%%%%%%%%%%%%
\section{My Model}

Supply and Demand are also be positively correlated. As demand increases supply will increase to match, but less than the quantity demanded, since the price should also be able to increase. This is in an idealized market of course, since if there are competitors, that could keep a company from raising there prices. However, a company will never supply more than the amount demanded since that will drive down prices and thus profit margin, while also increasing stocks leading to reduced profit in the future. When demand quantity decreases we would then expect the supply quantity to decrease more than this, for similar reasons to why it would increase less than with demand increase. We can call this amount of disconnect in change $\delta$ so that we have $\frac{\d q_S}{\d t} \approx (1-\delta)q_d$.

From macroeconomic textbooks we have that 
\begin{align}
	\dot{q}_S = &  c_1 q_D + (\frac{dS}{dq} - \frac{dD}{dq})^{-1}\dot{q}_D \\
	& \ \ \ +\eps\ddot{q}_D %\label{predict_demand}
\end{align}
But from this equation we come to the big question of what $\frac{dS}{dq}$ and $\frac{dD}{dq}$ mean. So perhaps it is still useful to call this term $(1-\delta)$ where we can theoretically understand $\delta$ as the inverse of the spread between what the producers and the consumers theoretically expect for that quantity of goods in the market. An interesting note is that for linear systems the amount that $\dot{q}_S$ changes does not depend on if the change in $q_D$ was positive or negative.

Further, is it possible that in an attempt to predict the necessary supply in the future the company adds on a second derivative term such as \eqref{predict_demand}? 
It would be optimizing in that if they expect things to change quickly, they can try not to run out of stock in stores. This may be especially relevant for the case of perishable goods where rather than always being behind demand until it reaches a equilibrium, trying to be ahead and always have supplied what the consumers will purchase. This would likely need an unknown constant in front of the term that is less than one (it probably has less cost associated to under-perform still). 

We also want to look into how the quantity demanded changes changes over time. I believe as a consumer, that this will be based mostly on the price, where $p$ is the transaction price taking place in the market. There is also something to be said for ease of access and rarity of the goods, so I believe that a positive quadratic term makes sense for the supply available effecting the quantity demanded. I argue that change in quantity supplied is not noticeable to a consumer. Either the store has the good or it does not, I have no idea what others are up to. Thus the change in quantity supplied term is left out. 

Something interesting that I always thought was left out is rarity. It is really obvious in trading cards, but also in Grey Poupon mustard. Something can be more expensive and that attracts people to it perhaps because of ideas of luxury. Because of this sort of thing, I do not believe that the interaction between supply and demand is linear, so then the question is what kind of non-linearity makes sense. 

\begin{equation}
		\dot{q}_D = - c_1 \dot{p} - c_2(q_S-c_3)^3
\end{equation}

In this system we can assume that companies may change their supply without a large effect on consumption, as long as the consumers can still easily access a good (or highly value the rarity of it which is handled in $c_3$).

\subsubsection{Taking Stocks of Goods into Account}
The stock of excess supplied goods also needs to be taken into account. This is only relevant for the case of non-perishable goods, just as I would guess that the second derivative term is only relevant for the case of perishable goods. Do companies want at least a certain amount stockpiled? The only difference in the model is that companies would make less money, so we can ignore this in terms of understanding the dynamics. One major difference is that the as stockpiles grow, prices tend to drop as a way of clearing them out. So it's not just about the amount supplied, but the amount in total available that determines changes in price. We definitely know that 
\begin{equation*}
	\dot{b} = q_s - q_d
\end{equation*}



%%%%%%%%%%%%%%%%%%%%%%%%%%%%%%%%%%%%%%%%%%%%%%%%%%%%%%%%%%%%%%%%%%%%%%%%%%%%%%%%%%%%%%%%%%%%%%%%%%%%%%%%%%%%%%%%%%%%%%%%%%%%%%%%%%%%%%%%
\section{Problems}
\begin{itemize}
	\item It seems like it would be less relevant/useful to model q, the surplus quantity supplied, than changes in the quantity demanded and the quantity supplied. This is partially because then when we see changes (since presumable preferences do have the possibility of changing over time) we can relate this to the model better.
	\item I don't think S and D should be rates but functions that are greater than or equal to 0. This way aggregate supply and aggregate demand could also be modeled in the same system, and these are generally not considered to be linear.
\end{itemize}


\end{document}